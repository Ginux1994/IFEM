% $Id: sim-input.tex,v 1.3 2010-08-18 11:23:00 kmo Exp $

\documentclass{article}
\usepackage{color}
\usepackage{/home/kmo/LaTeX/Sty/a4}
\input{/home/kmo/LaTeX/Macros/NameList}

\title{{\large ICADA - Note 2010-001:}\hfill\mbox{}\\[2mm]
Specification of spline patch topology to SIM}
\author{Knut Morten Okstad and Trond Kvamsdal}

\def\cube{
\begin{picture}(95,85)
\thicklines
\put(15,15){\line(1,0){40}}
\put(15,50){\line(1,0){40}}
\put(15,15){\line(0,1){35}}
\put(55,15){\line(0,1){35}}
\put(15,50){\line(1,1){20}}
\put(55,50){\line(1,1){20}}
\put(55,15){\line(1,1){20}}
\put(35,70){\line(1,0){40}}
\put(75,35){\line(0,1){35}}
\put(15,15){\line(1,1){20}}
\put(35,35){\line(1,0){40}}
\put(35,35){\line(0,1){35}}
\thinlines
\put(10,10){\vector(-1,-1){7}}
\put(80,35){\vector(1,0){10}}
\put(35,75){\vector(0,1){10}}
\put( 0, 0){\mbox{I}}
\put(91,33){\mbox{J}}
\put(36,82){\mbox{K}}
\end{picture}}

\def\nodes{
\begin{picture}(60,55)
\put(20,20){\circle*{2}}
\put( 0, 0){\circle*{2}}
\put(60,20){\circle*{2}}
\put(40, 0){\circle*{2}}
\put(20,55){\circle*{2}}
\put( 0,35){\circle*{2}}
\put(60,55){\circle*{2}}
\put(40,35){\circle*{2}}
\end{picture}}

\def\edges{
\begin{picture}(60,55)
\put(10,10){\circle*{2}}
\put(50,10){\circle*{2}}
\put(10,45){\circle*{2}}
\put(50,45){\circle*{2}}
\put(35,20){\circle*{2}}
\put(20, 0){\circle*{2}}
\put(40,55){\circle*{2}}
\put(25,35){\circle*{2}}
\put(20,40){\circle*{2}}
\put( 0,20){\circle*{2}}
\put(60,40){\circle*{2}}
\put(40,15){\circle*{2}}
\end{picture}}

\def\faces{
\begin{picture}(60,55)
\put(40,40){\circle{5}}
\put(20,15){\circle{5}}
\put(10,25){\circle{5}}
\put(50,25){\circle{5}}
\put(30,10){\circle{5}}
\put(30,45){\circle{5}}
\end{picture}}

\begin{document}
\maketitle

\section{Assumptions}

We only allow for completely matching blocks for now.
That is, two adjacent blocks are topologically connected
with a common surface if, and only if, all control points (nodes)
defining the geometry of the surface are identical for the two blocks.
Otherwise, they are assumed not connected (we have a ``crack'').

\section{Input description, topology}

The simulation module must generate a unique global node number for all
control points in the model.
The input to this process should be as compact as possible, but should contain
sufficient information such that the node number generation can be carried out
locally within each patch in arbitrary order or in parallel.
Only topology information is included, all geometry data goes via the g2-files
from GoTools.

\begin{figure}[htb]
\begin{center}
\setlength{\unitlength}{4pt}
\begin{picture}(95,85)
\put( 0, 0){\cube}
\put(15,15){\nodes}
\put(31,35){\bf 1}
\put(11,14){\bf 2}
\put(78,35){\bf 3}
\put(58,13){\bf 4}
\put(31,70){\bf 5}
\put(11,49){\bf 6}
\put(78,69){\bf 7}
\put(58,48){\bf 8}
% Mark the <icnod> nodes
\put(32,31){{\Large\tt *} icnod1}
\put(72,31){{\Large\tt *} icnod2}
\put(24,66){icnod3 {\Large\tt *}}
\put(64,66){icnod4 {\Large\tt *}}
\put(39,34){\Large\tt *}\put(41,36){icnod5}
\put(19,14){\Large\tt *}\put(21,16){icnod6}
\put(39,69){\Large\tt *}\put(41,71){icnod7}
\put(19,49){\Large\tt *}\put(21,51){icnod8}
\put(35,40){{\Large\tt *} icnod9}
\put(5.8,20){icnod10 {\Large\tt *}}
\put(75,40){{\Large\tt *} icnod11}
\put(45.8,20){icnod12 {\Large\tt *}}
\end{picture}
\end{center}
\caption{Local vertex numbering convention $(\bullet)$
and the first node along each edge $(\star)$.}
\label{fig:vertices}
\end{figure}

\clearpage
For each patch, the input is as follows (see Figures~\ref{fig:vertices},
\ref{fig:edges} and~\ref{fig:faces} for the interpretation of the variables
ICNOD$i$, ISNOD$i$ and IINOD$i$):

\vskip2mm\hskip\parindent
\fbox{%
\begin{minipage}{8cm}
\begin{tabbing} XXXXXX \= XXXXXX \= XXX \= XXXXXX \= \kill
IBLOCK \\
IBNOD1 \> IBNOD2 \> ... \> IBNOD8 \\
ICNOD1 \> INCR1 \\
ICNOD2 \> INCR2 \\
... \\
ICNOD12 \> INCR12 \\
ISNOD1  \> INCI1 \> INCJ1 \\
ISNOD2  \> INCI2 \> INCJ2 \\
... \\
ISNOD6 \> INCI6 \> INCJ6 \\
IINOD1
\end{tabbing}
\end{minipage}}

\begin{figure}[htb]
\begin{center}
\setlength{\unitlength}{3.8pt}
\begin{picture}(95,85)
\put( 0, 0){\cube}
\put(15,15){\edges}
\put(25,28){\bf 1}
\put(69,23){\bf 2}
\put(25,63){\bf 3}
\put(69,58){\bf 4}
\put(50,37){\bf 5}
\put(37,17){\bf 6}
\put(57,73){\bf 7}
\put(41,52){\bf 8}
\put(32,54){\bf 9}
\put(10,34){\bf 10}
\put(79,54){\bf 11}
\put(59,29){\bf 12}
% Mark the <isnod> nodes
\put(38,37){{\Large\tt *} isnod1}
\put(17,17){\colorbox{white}{\Large\tt *}isnod2}
\put(24,35){isnod3 {\Large\tt *}}
\put(70.6,34){\colorbox{white}{\Large\tt *}}\put(64,36){isnod4}
\put(36,31){{\Large\tt *} isnod5}
\put(35,66){\colorbox{white}{\Large\tt *}isnod6}
\end{picture}
\end{center}
\caption{Local edge numbering convention $(\bullet)$
and the first node on each surface $(\star)$.}
\label{fig:edges}
\end{figure}

\begin{namelist}{IBLOCK}
\item[IBLOCK] Spline patch index
\item[IBNOD$i$] Global node number of vertex $i$
\item[ICNOD$i$] Global node number of second point along edge $i$
\item[INCR$i$] Increment in global numbering along the edge ($\pm1$)
\item[ISNOD$i$] Global node number of first interior point on face $i$
\item[INCI$i$] Increment in global numbering in local $I$-direction on the face ($\pm1$)
\item[INCJ$i$] Increment in global numbering in local $J$-direction on the face ($\pm1$)
\item[IINOD1] Global node number of the first interior point of the patch
\end{namelist}

The local $I$ and $J$ directions for a face are defined as the two remaining
directions when removing the index defining the normal direction of that face
from the $I-J-K$ triplet.
That is, for local faces 1 and 2, the local $I-J$ directions correspond the
``global'' $J-K$ directions (depicted in Figure~\ref{fig:faces}).
For local faces 3 and 4, the local $I-J$ directions correspond to the ``global''
$I-K$ directions, respectively, whereas for local faces 5 and 6 they coincide
with the global $I-J$ directions.

\begin{figure}[htb]
\begin{center}
\setlength{\unitlength}{3.8pt}
\begin{picture}(95,85)
\put(0,0){\cube}
\put(15,15){\faces}
\put(55,54){\bf 1}
\put(35,29){\bf 2}
\put(25,39){\bf 3}
\put(65,39){\bf 4}
\put(45,24){\bf 5}
\put(45,59){\bf 6}
% Mark the <iinod> nodes
\put(37.2,32.8){\colorbox{white}{\Large\tt *}}
\put(40,36){iinod1}
\end{picture}
\end{center}
\caption{Local face numbering convention $(\circ)$
and the first interior node $(\star)$.}
\label{fig:faces}
\end{figure}

\section{Properties and boundary conditions}

All physical properties defined in the GPM-module are mapped onto the
spline model through a set of user-defined codes.
The actual interpretation of each code is defined within SIM itself,
via a separate input file.
The property codes are specified through the following syntax:

\vskip2mm\hskip\parindent\fbox{PCODE~~~IBLOCK~~~LDIM~~~LINDEX}

\begin{namelist}{IBLOCK}
\item[PCODE]  Property or boundary condition code
(either a string or an integer value)
\item[IBLOCK] Spline patch index
\item[LDIM]   Local entity dimension flag (0, 1, 2, or 3)
\item[LINDEX] Local entity index which is assigned the property
\begin{itemize}
\item Local vertex if LDIM = 0
\item Local edge if LDIM = 1
\item Local face if LDIM = 2
\item Not referenced if LDIM = 3
\end{itemize}
\end{namelist}

The local ordering of the vertices, edges and faces follows the convention
defined in Figures~\ref{fig:vertices}, \ref{fig:edges} and~\ref{fig:faces},
respectively.
\end{document}
